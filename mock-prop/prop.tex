\documentclass[a4paper,12pt]{article}

\usepackage[utf8]{inputenc}
\usepackage[T1]{fontenc}
\usepackage{textcomp}
\usepackage{parskip}

\title{Feature Boosting in Multiomics}
\author{Michal Grochmal
  $<$\href{mailto:grochmal@member.fsf.org}{grochmal@member.fsf.org}$>$
}
\date{\today}

\usepackage[colorlinks=true]{hyperref}

\begin{document}
\maketitle
\thispagestyle{empty}

Proteomics and genomics data is very difficult to classify programaticaly.
Although multiomics databases posses huge amounts of data in structured schemas
we cannot be sure that the structure of these databases actually represent
features of the data.  The major example that the structure of the multiomics
databases do not represent all features in the proteomic or genomic data is the
fact that between distinct databases there are enormous differences in the
hierarchies used to structure the data.

We will try to prove the fact that the schemas of the multiomic databases
represent only a small part of the features that can be found in the multiomic
data.  If we can find a big amount of features in the multiomic data we can
evaluate the correlation and importance of these features to select the
features most relevant for the classification.  Using only the selected
features shall bestow the accuracy of the classification.  To perform this we
will use deep learning techniques over flattened data from multiomics
databases.

\end{document}

