\documentclass[11pt,a4paper,twoside,openright]{report}

\usepackage[utf8]{inputenc}
\usepackage[T1]{fontenc}
\usepackage{lmodern}
\usepackage{textcomp}


\usepackage{parskip}
\usepackage{fancyhdr}

\title{Multi'omics analysis: three month report}
\author{Michal Grochmal
  $<$\href{mailto:grochmal@member.fsf.org}{grochmal@member.fsf.org}$>$\\
  \\
  Department of Electronic Engineering and Computer Science\\
  Queen Mary University of London
}
\date{\today}

\usepackage[ colorlinks=true
           , plainpages=false
           , pdfpagelabels
           , pageanchor=false ]{hyperref}

\begin{document}
\maketitle

\newpage
\null
\thispagestyle{empty}
\newpage


\pagestyle{fancy}
\lhead{}
\chead{MULTI'OMICS ANALYSIS: THREE MONTH REPORT}
\rhead{}

\begin{abstract}
Experimental data of a single set of runs of one experiment can prove tricky to
interpret, yet in most single experiments we expect a result confirming or
denying a predefined hypothesis.  Building interpretations on data from several
experiments at once faces extra issues: huge dimensionality if we just
concatenate the data, missing data in different places across the different
experiments and correlation of the measures provided by each experiment.
Classical and recent methods to integrate data from several sources may prove
effective to conquer these difficulties.  We evaluate known methods of scaling,
clustering and regression in the task of integrating such sources.
\begin{flushright}
\emph{Similia similibus curantur -- Paracelsus}
\end{flushright}
\end{abstract}

\newpage
\null
\thispagestyle{empty}
\newpage

\newpage
\setcounter{page}{1}
\pagenumbering{arabic}

% clever trick to clear the pages before chapter openings
\clearpage{\pagestyle{empty}\cleardoublepage}
\chapter{Introduction}

\end{document}

