\documentclass[hyperref={colorlinks=true}]{beamer}

\usepackage[utf8]{inputenc}
\usepackage[T1]{fontenc}
\usepackage{lmodern}
\usepackage{textcomp}

\usepackage{multirow}

\mode<presentation>
{
  \usetheme{Warsaw}
  \usecolortheme{seahorse}
  \usecolortheme{orchid}
  \setbeamercovered{transparent}
}

\title{MULTI'OMICS ANALYSIS \\ three month report}
\author{Michal Grochmal
  <\href{mailto:grochmal@member.fsf.org}{grochmal@member.fsf.org}>}
\institute{Queen Mary University of London}
\date{December 2014}
\subject{Data Integration}

\begin{document}

\begin{frame}
  \titlepage
\end{frame}

\begin{frame}{Outline}
  \tableofcontents[pausesections]
\end{frame}

\section{Data Integration}

\subsection{Why is it useful?}
\begin{frame}{Why is it useful?}
  \begin{itemize}
    \item Classifiers may need to be built for problems for which data has not
          a single well defined source
    \item What we do with the data before we apply the classification can vary
    \item Normalisation can be a (super simple) example of how operations
          before running the classifiers influence results
    \item In general we use integration techniques to improve classification
          accuracy
  \end{itemize}
\end{frame}

\subsection{Multiple data points}
\begin{frame}{Cars and Motorbikes}
  \begin{block}{Car data}
    \begin{tabular}{l|r|rr}
    Car & Sport & Weight (kg) & Wheel diameter (mm) \\
    \hline
    Mazda 1.5 & 0 & 1550 & 730 \\
    Subaru 20 & 1 & 2100 & 832 \\
    \hline
    \end{tabular}
  \end{block}
  \begin{block}{Motorbike data}
    \begin{tabular}{l|r|rr}
    Motorbike & Sport & Weight (kg) & Wheel diameter (mm) \\
    \hline
    Kawasaki   & \multirow{2}{*}{0}
               & \multirow{2}{*}{550}
               & \multirow{2}{*}{912} \\
    Marauder   & & & \\
    Java 350cc & 1 &  312 & 1010 \\
    \hline
    \end{tabular}
  \end{block}
\end{frame}

\begin{frame}{Integration by conatenation}
  \begin{block}{Vertical concatenation $(\cup)$}
    \begin{tabular}{l|r|rr}
    Vehicle & Sport & Weight (kg) & Wheel diameter (mm) \\
    \hline
    Mazda 1.5   & 0 & 1550 & 730 \\
    Subaru 20   & 1 & 2100 & 832 \\
    Kawasaki M. & 0 &  550 & 912 \\
    Java 350cc  & 1 &  312 & 1010 \\
    \hline
    \end{tabular}
  \end{block}
  \begin{block}{Horizontal concatenation $(\pi)$}
    \begin{tabular}{l|r|rrrr}
    Vehicle & Sport & Car W & Car WD & Bike W & Bike WD \\
    \hline
    Mazda 1.5   & 0 & 1550 & 730 &   - &    - \\
    Subaru 20   & 1 & 2100 & 832 &   - &    - \\
    Kawasaki M. & 0 &    - &   - & 550 &  912 \\
    Java 350cc  & 1 &    - &   - & 312 & 1010 \\
    \hline
    \end{tabular}
  \end{block}
\end{frame}

\subsection{Multiple data sources}
\begin{frame}{Sport cars}
  \begin{columns}[c]
  \begin{column}{.5\textwidth}
    \begin{block}{Data from the manufacturer}
      \begin{tabular}{l|r|rr}
      Car & Sport & W & WD \\
      \hline
      Mazda  & 0 & 1550 & 730 \\
      Subaru & 1 & 2100 & 832 \\
      \hline
      \end{tabular}
    \end{block}
    \begin{block}{Data from evaluation website}
      \begin{tabular}{l|r|rr}
      Car & Sp & MPG & Power \\
      \hline
      Mazda  & 0 & 15 &  730 \\
      Subaru & 1 & 10 & 1280 \\
      \hline
      \end{tabular}
    \end{block}
  \end{column}
  \begin{column}{.5\textwidth}
    \begin{block}{Data from forum on cars}
      \begin{tabular}{l|r|rr}
      Car & Sp & Reliab. & Drive \\
      \hline
      Mazda  & 0 & 6.3 & 7.9 \\
      Subaru & 0 & 7.2 & 7.9 \\
      \hline
      \end{tabular}
    \end{block}
  \end{column}
  \end{columns}
\end{frame}

\begin{frame}{Vertical concatenation}
  \begin{block}{Car data}
    \begin{tabular}{l|r|rrrrrr}
    Car & Sport & W & WD & MPG & Power & Reliab. & Drive \\
    \hline
    Mazda  & 0 & 1550 & 730 & 15 &  730 & 6.3 & 7.9 \\
    Subaru & 1 & 2100 & 832 & 10 & 1280 & 7.2 & 7.9 \\
    \hline
    \end{tabular}
  \end{block}
  \pause
  Horizontal concatenation does not make sense in this case.

  This is a form of \emph{early integration}.

\end{frame}

\section{Classification of data integration}

\subsection{General}
\begin{frame}{Classification by concept}
  \begin{description}
    \item[Early integration] \hfill \\
    \pause
    \begin{itemize}
      \item Concatenates data before processing
      \item Happens before classification
    \end{itemize}
    \pause
    \item[Late integration] \hfill \\
    \pause
    \begin{itemize}
      \item Classifies data before integration
      \item Integration happens by combining classifier output
      \item e.g. majority vote or cummulative vote
    \end{itemize}
    \pause
    \item[Intermediate integration] \hfill \\
    \pause
    \begin{itemize}
      \item More complex than early and late integration
      \item Combines the data assigning \emph{weights} to measures
      \item There are several ways to assign such weights
    \end{itemize}
  \end{description}
\end{frame}

\subsection{Techniques}
\begin{frame}{Classification by technique}
  Early and late integrations are rather simple, therefore we discuss
  intermediate integration for here on.
  \pause
  \begin{itemize}
    \item Clustering (Parallel clustering)
    \item Euclidean distance (MDS)
    \item Graph techniques (SNF, SC)
    \item Bayesian Networks (Bayesian correlation)
    \item Vectorial space (PLS)
  \end{itemize}
\end{frame}

\section{Work plan}

\subsection{Multi'omics}
\begin{frame}{Multi'omics}
  \begin{itemize}
    \item Lots of biological data, e.g. TCGA
    \item AI problems with multi'omics
      \begin{itemize}
        \item Feature extraction process is much more needed than data
              integration in multi'omics
        \item AI cannot solve a problem that a human cannot solve
      \end{itemize}
    \pause
    \item Feature extraction need to be done in possession of some expert
          knowledge of the area
    \item Visualisation techniques may help in this
    \item AI will not solve the lack of understanding in the area, AI can be
          used to prove/reject specific hypotheses in the area
  \end{itemize}
\end{frame}

\subsection{Plans}
\begin{frame}{Some possible outlines}
  \begin{columns}[T]
  \begin{column}{.5\textwidth}
    \begin{block}{Plan A}
      \begin{itemize}
       \item[Jan] SNF and MDS on GBM
       \item[Feb] Graph dimensions in both techniques
       \item[Mar] ...
      \end{itemize}
    \end{block}
    \begin{block}{Plan C (example)}
      \begin{itemize}
       \item[Jan] ELF header
       \item[Feb] Test environment and prototype build
       \item[Mar] Test the pototype and some simple bindings
      \end{itemize}
    \end{block}
  \end{column}
  \begin{column}{.5\textwidth}
    \begin{block}{Plan B}
      \begin{itemize}
       \item[Jan] Distributed SNF diagram
       \item[Feb] Implementation and testing of distribution
       \item[Mar] Failure resistence and optimisation
       \item[Apr] Documentation
      \end{itemize}
    \end{block}
  \end{column}
  \end{columns}
\end{frame}

\subsection{Current work}
\begin{frame}{Outline of a workplan design}
\begin{description}
  \item[What?] \hfill \\
  \begin{itemize}
    \item Problem
    \item How to measure it
    \item If we cannot measure it, get a different problem
  \end{itemize}
  \pause
  \item[Why?] \hfill \\
  \begin{itemize}
    \item Why a solution to this problem is desirable
  \end{itemize}
  \item[How?] \hfill \\
  \begin{itemize}
    \item This is easy
  \end{itemize}
\end{description}
\end{frame}

\end{document}

